\documentclass[10pt]{article}
    \usepackage[utf8]{inputenc}
    \usepackage{amsmath}
    \usepackage[margin=0.5in]{geometry}
    \usepackage{graphicx}
    \usepackage{caption}

    \title{Message Passing Interface and Parallelism}
    \author{David Sharp : ds16797 : Candidate 36688}
    \date{December 2018}

    \begin{document}
    \maketitle
    \section{Final Implementation}
    My final implementation that I coded with MPI for the 5-point stencil task distributes
    columns out between processes - leaving any remainder with the highest rank process - and
    implements a 2-D halo-exchange cyclicly using blocking combined send/receive calls.
    It then copies the answer from the write-too array ($curImage$) to the read-from array
    ($preImage$) and repeats the iteration for the required 200 times in the test cases.
    It checks to see whether the inputted image size has more rows than columns and makes
    sure it distributes over the larger by swapping the array dimensions if necessary.

    This is what occurs in the case that the size of image is comparatively small,
    when the image is larger instead I ignore the answer copying and instead just do two
    iterations effectively `in-place` similar to the serial stencil code. I think due to the
    higher overhead of copying an array at higher sizes this strategy performs better.

    After the stencil is complete, the master rank takes charge of taking back in the answers
    from the `student` processes. It allocates space for an array to store each of the student's
    answers in, then places its own answers in the array and sets up to receive each student's answers
    column by column and place them in the same array. It then feeds the array into the serial
    image output function. This outputting process takes by far the bulk of the actual runtime of
    the program, particularly for very large images, though not measured for this assignment.

    \subsection{Implementation Decisions}
    There were several implementation paths where the results surprised me, the first being that
    I assumed that my method for distributing rows out to the processes was flawed in that the
    last row would end up with more work than the rest of the processes and would thus slow the
    entire program down due to synchronous communication throughout the program. I made the change
    to make it so that since the remainder is at most one less than the number of processes in the
    world, I could just give each process (starting with the highest ranked process and iterating downwards)
    an additional row to manage. However, this implementation slowed down my code very slightly, on further
    thought even if it had sped up my code it would likely not have been a significant improvement since the
    imbalance in rows between ranks is at most one less than the number of ranks due to it being
    $Disparity = Number of Columns (Mod Number of processes)$
    This means that this disparity becomes less of an issue as the size of the image scales, and becomes
    more of an issue as the number of processes scales. These two factors together mean that this change would
    be a improvement if we were running relatively small images in 2019 on a double AMD Rome node, however
    with 16 core nodes on Bluecrystal3, scaling with the number of processes on a single node is not a
    useful change.

    As for the change in strategy based on image size, I don't honestly quite understand why stencilling back
    and forth rather than copying the array is slower for small images since it seems to just be fewer operations.
    I imagine that there is some compiler optimisation that gets removed when the for loop is expanded to include
    two send/receive halo exchanges. However for images of around size 3000x3000, doing the double stencil starts
    becoming significantly quicker, at size 8000x8000 the stencil runs in nearly half the time when not copying the
    image array.

    \section{Timings}
    \subsection{Timings with Array Copying}

    \begin{center}
        \begin{tabular}{ |c||c|c|c|c|c|c|c|c|c|c|c|c|c|c|c|c| }
            \multicolumn{17}{| c |}{Stencil time\/s 3sf (Mean of Three Runs)}
            \hline
            Processes & 1 & 2 & 3 & 4 & 5 & 6 & 7 & 8 & 9 & 10 & 11 & 12 & 13 & 14 & 15 & 16 \hline
            1024x1024 & 0.534 & 0.256 & 0.167 & 0.125 & 0.103 & 0.0852 & 0.0730 & 0.0689 & 0.0608 & 0.0527 & 0.0506 & 0.0475 & 0.0446 & 0.0418 & 0.0392 & 0.0373 \hline
            4096x4096 & 10.6 & 5.56 & 3.77 & 2.88 & 2.69 & 2.77 & 2.37 & 2.14 & 2.29 & 2.06 & 2.25 & 2.18 & 2.25 & 2.14 & 2.25 & 2.11 \hline
            8000x8000 & 35.5 & 17.3 & 13.1 & 9.86 & 9.89 & 10.4 & 8.93 & 7.82 & 8.59 & 7.73 & 8.45 & 7.86 & 8.45 & 7.84 & 8.38 & 7.90 \hline
        \end{tabular}
    \end{center}

    \subsection{Timings with `in-place` Double Stencil}

    \begin{center}
        \begin{tabular}{ |c||c|c|c|c|c|c|c|c|c|c|c|c|c|c|c|c| }
            \multicolumn{17}{| c |}{Stencil time\/s 3sf (Mean of Three Runs)}
            \hline
            Processes & 1 & 2 & 3 & 4 & 5 & 6 & 7 & 8 & 9 & 10 & 11 & 12 & 13 & 14 & 15 & 16 \hline
            1024x1024 & 1.09 & 0.543 & 0.366 & 0.276 & 0.227 & 0.189 & 0.163 & 0.144 & 0.133 & 0.121 & 0.111 & 0.104 & 0.104 & 0.0919 & 0.0879 & 0.0807  \hline
            4096x4096 & 14.3 & 7.43 & 4.73 & 3.83 & 2.94 & 2.54 & 2.17 & 2.02 & 1.76 & 1.61 & 1.49 & 1.38 & 1.36 & 1.29 & 1.27 & 1.19  \hline
            8000x8000 & 31.8 & 20.0 & 13.0 & 12.1 & 10.9 & 7.18 & 7.10 & 6.25 & 5.16 & 4.64 & 4.64 & 4.27 & 4.45 & 4.28 & 4.36 & 4.12\hline
        \end{tabular}
    \end{center}

    \subsection{Timing Analysis}
    These timings, especially the comparison between the two methods, show some interesting behaviour; clearly there is a significant
    overhead to the different strategies, looking at the timings for the 4096 image it's not until we start running six processes that
    we see any performance increase over the array copying approach. Also notably, the `in-place` method performs nearly half as well
    as array copying for the 1024 image.
    Once we start using enough processes to stencil on a large enough image, we start seeing very significant performance increases with the `in-place` approach,
    I was surprised to see that more or less throughout it outperformed the other approach on the 8000x8000 image, I had expected the rate-limiting
    factor of the memory bandwidth usage to mean that the 16 core performance wouldn't be twice as fast as the array copying approach like for the 4096x4096 image.

    \section{Code/Timing Evaluation}
    It's no great surprise that at the high end of process numbers, the code is heavily memory bandwidth bound. As can be seen from the timings
    relative to the time for one process in that size category, while as number of processes increases initially
    $Relative Time for n processes \approx \frac{1}{n}$ however when we get to using enough processes, this is no longer true. From my understanding
    this is due to - for the small image, or for small numbers of processes - being more comparatively compute bound, hence our times increasing
    proportionally with the number of processes which mainly provide more flops and less so memory bandwidth.
    This is most clear when looking at the times for the 8000x8000 image, while the first three or four process number increases offer us a significant
    performance boost when we start getting to using about eight or nine processes we have already hit near enough the time when running the code with 16 processes.
    I believe this specific eight or nine breakpoint is due to the specific architecture of Bluecrystal3, since Bluecrystal3 nodes contain two eight core sockets,
    once we start needing nine processes we are definitely using both sockets and hence at least have access to the maximum possible STREAM bandwidth.
    Since I am still a factor of two off the benchmarks for all the images that there is more I could have done to utilise more memory bandwidth.

    \subsection{Theoretical Maximums}
    In the case that we assume our message passing has no cost associated, i.e we reduce our code to cost 10 FMAs per pixel per iteration, we can obtain the
    following table for Bluecrystal3 theoretical maximum speeds on the stencil code, using the Blackboard roofline model.

    \begin{center}
        \begin{tabular}{ | c || c | c | c |}
          \hline
          Image Size & Giga FMAs required & 1 core timing/s & 16 core timing/s \hline
          1024x1034 & 2.1 & 0.05 & 0.0032 \hline
          4096x4096 & 33 & 0.79 & 0.05 \hline
          8000x8000 & 128 & 3.08 & 0.19 \hline
        \end{tabular}
    \end{center}

    The first takeaway from the table is that message passing has a very real cost, the code is severely bandwidth bound,
    if we could manage to just ram the entire image array onto 16 cores uncaringly and suffer no errors or downsides we could reap a 10-20x speedup on my code.
    The second takeaway which is somewhat unfortunate for me, is that the closest I get to the theoretical maximums is on the 8000x8000 image on a single core, where
    I'm \textit{only} a factor of 10 off. This means that my code is scaling badly in terms of the cost of message passing.

    \subsection{Improvements}
    Consider my message passing(MP) scheme mathematically with processes p, and image dimensions x (width) and y (height).

    The cost of passing messages is the cost of the left cyclic pass plus the cost of the right cyclic pass, $MP Cost = MP Left + MP Right$
    Assuming the number of processes divides the width of the image, each process is managing $\frac{x}{p}$ columns and each row extends to
    the depth of the image.
    $MP Left \propto py , MP Right \propto pxy hence MP Cost \propto 2py$

    This means that the amount of message passing we have to do scales linearly with the number of processes and the size of the image.

    If we instead consider a tiling approach we can say that $MP Cost = MP North + MP South + MP West + MP East$
    Assuming a square number of processes that divides the image dimensions, each process is managing a $\frac{x}{\sqrt{p}} by \frac{y}{\sqrt{p}}$ size tile.
    $$ MP North \propto \sqrt{p}\frac{x}{\sqrt{p}}
    MP North \propto x
    MP South \propto x
    MP West \propto \sqrt{p}\frac{y}{\sqrt{p}}
    MP West \propto y
    MP East \propto y

    MP Cost \propto 2x+2y$$

    Under the tiling scheme, with sufficiently sized tiles, the amount of message passing is completely independent of the amount of processes we are using.
    More realistically, as the number of tiles increases on an image there comes a point where all we are doing is passing messages, but the number of processes
    can be selected for to counter this.












  \end{document}
